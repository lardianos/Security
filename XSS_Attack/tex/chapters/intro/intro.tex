% Writed by: Stavros Papantonakis
%
%!TEX TS-program = xelatex
%!TEX encoding = UTF-8 Unicode
%
\section*{Εισαγωγή}
%Aferesi kenou arxis paragrafou
\noindent
Το \textbf{scripting} μεταξύ ιστοτόπων (Cross-Site scripting ή XSS) είναι
ένας τύπος ευπάθειας που συναντάται συνήθως στις web εφαρμογές. Αυτή η ευπάθεια
επιτρέπει στους εισβολείς να εκτελούν κακόβουλο κώδικα (π.χ. προγράμματα σε
JavaScript) στον browser του θύματος.Χρησιμοποιώντας αυτόν τον κακόβουλο
κώδικα, οι επιτιθέμενοι μπορούν να κλέψουν τα διαπιστευτήρια ενός θύματος, 
όπως τα \textbf{session cookies}. Οι πολιτικές ελέγχου πρόσβασης που χρησιμοποιούν οι
browsers για την προστασία αυτών των διαπιστευτηρίων μπορούν να παρακάμπτονται
εκμεταλλευόμενες τις ευπάθειες XSS. Οι ευπάθειες αυτού του είδους μπορούν
δυνητικά να οδηγήσουν σε επιθέσεις μεγάλης κλίμακας.

\noindent
Για να κατανοήσουμε καλύτερα τι μπορούν να κάνουν οι επιτιθέμενοι εκμεταλλευόμενοι τις
ευπάθειες XSS, έχει δημιουργηθεί μια web εφαρμογή που ονομάζεται \textbf{Elgg} (στην
προκατασκευασμένη εικονική μηχανή Ubuntu 16.04). Το Elgg είναι μια πολύ δημοφιλής
εφαρμογή ανοικτού κώδικα για δημιουργία κοινωνικών δικτύων και έχει εφαρμόσει μια
σειρά αντιμέτρων για την αντιμετώπιση της απειλής XSS. Για να καταδείξουμε πώς
λειτουργούν οι επιθέσεις XSS, έχουμε απενεργοποιήσει αυτά τα αντίμετρα στον κώδικα της
εφαρμογής, καθιστώντας εκ προθέσεως το Elgg ευάλωτο σε επιθέσεις XSS. Χωρίς τα
αντίμετρα, οι χρήστες μπορούν να δημοσιεύσουν οποιοδήποτε αυθαίρετο μήνυμα,
συμπεριλαμβανομένων των προγραμμάτων JavaScript, στα προφίλ τους.

\noindent
Σε αυτό το εργαστήριο, οι φοιτητές πρέπει να αξιοποιήσουν αυτήν την ευπάθεια για να
ξεκινήσουν μια επίθεση XSS στο τροποποιημένο Elgg, με τρόπο παρόμοιο με αυτόν που
έκανε ο \textbf{Samy Kamkar} στο MySpace το 2005, μέσω του περιβόητου \textbf{Samy worm3}. Ο
απώτερος στόχος αυτής της επίθεσης είναι η διάδοση ενός XSS worm μεταξύ των χρηστών,
έτσι ώστε όποιος βλέπει ένα μολυσμένο προφίλ χρήστη να μολύνεται και ο ίδιος. Η ενέργεια
που εκτελεί ο κακόβουλος κώδικας είναι να μεταβάλει το προφίλ του θύματος και να
προσθέτει τον επιτιθέμενο στη λίστα φίλων του. Αυτό το εργαστήριο καλύπτει τα ακόλουθα
θέματα:
\begin{itemize}
	\item Cross-Site Scripting attack
	\item XSS worm and self-propagation
	\item Session cookies
	\item HTTP GET and POST requests
	\item JavaScript and Ajax
\end{itemize}
Χρισημα links:
\begin{itemize}
	\item[1] \textbf{XSS} \url{http://projects.webappsec.org/w/page/13246920/Cross%20Site%20Scripting}
	\item[2] \textbf{Elgg} \url{https://elgg.org}
	\item[3] \textbf{Sumy worm} \url{https://en.wikipedia.org/wiki/Samy_(computer_worm)}
\end{itemize}