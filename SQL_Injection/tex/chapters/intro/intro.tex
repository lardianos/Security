% Writed by: Stavros Papantonakis
%
%!TEX TS-program = xelatex
%!TEX encoding = UTF-8 Unicode
%
\section*{Εισαγωγή}
%Aferesi kenou arxis paragrafou
\noindent
Η SQL injection είναι μια τεχνική έγχυσης κώδικα που εκμεταλλεύεται τις ευπάθειες
στην αλληλεπίδραση μεταξύ \textbf{web εφαρμογών}και \textbf{διακομιστών βάσεων δεδομένων}.
Η ευπάθεια εμφανίζεται όταν τα δεδομένα που εισάγει ο χρήστης δεν ελέγχονται
σωστά μέσα στις εφαρμογές ιστού, πριν να σταλούν στους διακομιστές βάσης
δεδομένων του back-end.

\noindent
Πολλές εφαρμογές web λαμβάνουν δεδομένα από χρήστες και στη συνέχεια
χρησιμοποιούν αυτές τις εισόδους για την \textbf{κατασκευή ερωτημάτων SQL}, ώστε να
μπορούν να λάβουν πληροφορίες από τη βάση δεδομένων. Οι εφαρμογές Web
χρησιμοποιούν επίσης ερωτήματα SQL για την αποθήκευση πληροφοριών στη βάση
δεδομένων. Αυτές είναι κοινές πρακτικές στην ανάπτυξη εφαρμογών ιστού. Όταν τα
ερωτήματα SQL \textbf{δεν είναι προσεκτικά} κατασκευασμένα, μπορεί να προκύψουν
τρωτά σημεία SQL injection. Η SQL injection είναι μια από τις πιο κοινές επιθέσεις
στις εφαρμογές ιστού.

\noindent
Σε αυτό το εργαστήριο, θα χρησιμοποιήσουμε μια \textbf{εφαρμογή ιστού} που είναι
ευάλωτη στην προσβολή SQL Injection. Η εφαρμογή αυτή περιλαμβάνει τα κοινά
σφάλματα που γίνονται από πολλούς προγραμματιστές ιστού. Ο στόχος των
φοιτητών είναι να βρουν τρόπους εκμετάλλευσης των τρωτών σημείων SQL Injection,
να επιδείξουν τη ζημιά που μπορεί να επιφέρει η επίθεση αυτή και να γνωρίσουν τις
τεχνικές που μπορούν να βοηθήσουν στην αποφυγή αυτών των επιθέσεων. Αυτό το
εργαστήριο καλύπτει τα ακόλουθα θέματα:
\begin{itemize}
	\item SQL statement: SELECT and UPDATE statements
	\item SQL injection
	\item Prepared statement
\end{itemize}